
\begin{frame}{Contents}
   \begin{itemize}
       \item[\textcolor{red}{$\checkmark$}] Chisel の導入
       \item[\textcolor{red}{$\checkmark$}] Pros. \& Cons.
       \item[\textcolor{red}{$\rhd$}] ChiselImProc の説明
       \item Vivado HLS との比較
   \end{itemize} 
\end{frame}



\begin{frame}{ChiselImProc の全体構成}
    \centering
    \begin{tikzpicture}
        \draw (-5.1,-2.5) rectangle (0.1,2.5);
        \fill[cyan!20] (-5,1.1) rectangle (0, 1.7);
        \fill[red!10] (-5,0.4) rectangle (0, 1.0);
        \foreach \x in {3,2,1,0,-1,-2,-3}
        {
            \draw (-5,{0.7*\x-0.3}) rectangle (0,{0.7*\x+0.3});
        }
        \node at (-2.5, 2.1) {RGB2GrayFilter};
        \node at (-2.5, 1.4) {GaussianFilter};
        \node at (-2.5, 0.7) {SobelAndNonMaxSupression};
        \node at (-2.5, 0) {ZeroPadding};
        \node at (-2.5, -0.7) {HystThreshold};
        \node at (-2.5, -1.4) {HystThresholdComp};
        \node at (-2.5, -2.1) {Gray2RGBFilter};
        \foreach \x in {2,1,0,-1,-2,-3}
        {
            \draw[-Stealth] (0, {0.7*\x+0.7}) to [bend left=60] node[right] {8 bits} (0, {0.7*\x});
        }
        
        \draw[-Stealth] (-2.5, 2.9) node[above] {24 bits} -- (-2.5, 2.5);
        \draw[-Stealth] (-2.5, -2.5) -- (-2.5, -2.9) node[below] {24 bits};
    \end{tikzpicture}
    \begin{itemize}
        \item \textcolor{cyan}{
             Gaussian Filter は8bitの5x5サイズの畳み込みで遅延。 }
        \item \textcolor{cyan}{
            3x3 サイズに落として計算。}
    \end{itemize}
\end{frame}



\begin{frame}{SobelAndNonMaxSupression の構成}
    \begin{columns}[t,onlytextwidth]
        \begin{column}{0.5\textwidth}
        SobelAndNonMaxSupression の構成
    \centering
    \begin{tikzpicture}
        \draw (-4.5,-0.8) rectangle (-0.5, 0.8);
        \fill[red!10] (-4.4, 0.05) rectangle (-0.6, 0.65);
        \foreach \x in {-1,1}
        {
            \draw (-4.4, {0.35*\x-0.3}) rectangle (-0.6, {0.35*\x+0.3});
        }
        \node at (-2.5, 0.35) {SobelFilter};
        \node at (-2.5, -0.35) {NonMaxSupression};
        \draw[-Stealth] (-0.6, 0.35) to [bend left=60] node[right] {GradPix} (-0.6,-0.35);  
        
        \draw[-Stealth] (-2.5, 1.5) node[above] {8 bits} -- (-2.5, 0.8);
        \draw[-Stealth] (-2.5, -0.8) -- (-2.5,-1.5) node[below] {8 bits};
   \end{tikzpicture} 
   \end{column}
   \begin{column}{0.5\textwidth}
        SobelFilterの構成
       \centering
       \begin{tikzpicture}
           \draw (-4.5, -1.1) rectangle (-0.5, 1.1);
           \only<1>{
                \fill[cyan!20] (-4.4, -0.3) rectangle (-0.6, 0.3);
           }
           \only<2>{
                \fill[olive!20] (-4.4, 0.4) rectangle (-0.6, 1.0);
           }
           \foreach \x in {-1,0,1}
           {
                \draw (-4.4, {0.7*\x-0.3}) rectangle (-0.6, {0.7*\x+0.3});
           }
           \node at (-2.5, 0.7) {SobelConvolution};
           \node at (-2.5, 0) {SqrtWrapper};
           \node at (-2.5, -0.7) {CalculateGradient};
           
           \draw[-Stealth] (-2.5, 1.9) node[above] {8bits} -- (-2.5, 1.1);
           \draw[-Stealth] (-2.5,-1.1) -- (-2.5, -1.9) node[below] {GradPix};
       \end{tikzpicture}
   \end{column}
   \end{columns}
   \only<1>{
        \textcolor{cyan}{
            32 bit の開平法は遅延するので、レジスタを導入。
        }
   }
   \only<2>{
        \textcolor{olive}{
            16 bit の3x3サイズの畳み込みが遅い可能性。
        }
   }
\end{frame}